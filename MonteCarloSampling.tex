\begin{comment}
	\pagebreak
\end{comment}

\section{Monte Carlo Sampling}
$p(y^* | x^*, x_{1:n}, y_{1:n}) = \int p(y^* | x^*, \theta) p(\theta | x_{1:n}, y_{1:n}) = \\
\mathbb{E}_{\theta \sim p(.|X,y)}[p(y^*|x^*,\theta)]
\approx \frac{1}{m}\Sigma^m_i p(y^*|x^*,\theta^{(i)}), \theta^{(i)} \sim p(\theta|X,y)$\\
\textbf{LargeNums:} $\mathbb{E}_P[f(X)] \approx \frac{1}{N}\sum_i^N f(x_i)$ \textbf{Hoeffding:} f in $[0, C]$, 
$P(|\mathbb{E}_P[f(X)] - \frac{1}{N}\sum_i^N f(x_i)| > \epsilon) \leq 2 \exp(-2N\epsilon^2 / C^2)$\\
\begin{comment}
	\textbf{Intuition:} Based on the law of large numbers, we can derive the required number of samples N to be in a certain probability bound.\\
\end{comment}

\textbf{Approx:} $\frac{1}{Z}Q(x) = P(x)$, MC seq. $X_{1:N}$ with stationary $P(x)$\\
Prior $P(X_1)$, transitions $P(X_{t+1}|X_t)$ i.i.d of $t$; each row sums 1 \\
iff exists t; $\forall x, x'$; $p^{(t)}(x | x') > 0$
\\
\begin{comment}
	The Markov chain is a sequence of Random Variables, mostly the random variables are states in $\{1,..,M\}$.\\
	The MC is stationary if the transition probabilities are independent of t, e.g. they don't change if t changes.\\
\end{comment}

\textbf{Ergodic:} Reach all states from everywhere in exactly t steps\\
\begin{comment}
	\textbf{Note:} It must be exactly t steps to reach every state from everywhere. 
	\textbf{Example:} MC (1)-(2), whereas the state (2) can be reached from (1) after t=uneven steps, whereas it can be reached from (2) only after t=even steps.\\
\end{comment}

\textbf{Statnary:} $\lim_{\infty} P(X_N = x) = \pi(x)$, impl by$\uparrow$, $1 \cdot \pi^T = P^T \pi^T$\\

\textbf{Sample:} $x_1 \sim P(X_1), x_n \sim P(X_N | X_{N-1} = x_{N-1})$\\

\textbf{Goal:} $\pi(x) = \frac{1}{Z} Q(x)$, need to specify $P(x|x')$\\

\textbf{Balance:} $\frac{1}{Z}Q(x)P(x'|x) = \frac{1}{Z}Q(x')P(x|x') \Rightarrow \pi(x) = \frac{1}{Z}Q(x)$\\
\begin{comment}
Given unnormalized distribution Q(x), we want to design a Markov chain with stationary distribution $\pi$. We can ensure that the stationary distribution is correct if it satisfies the detailed balance equation.\\
	It suffices to show that $P(X_t = x) = \frac{1}{Z}Q(x) \Rightarrow P(X_{t+1} = x) = \frac{1}{Z}Q(x)$.\\ 
	Proof: $P(X_{t+1} = x) = \sum_{x'} P(X_{t+1} = x, X_t = x') \\
	= \sum_{x'} P(X_{t+1} = x| X_t = x') P(X_t = x') \\
	= \sum_{x'} P(X_{t+1} = x| X_t = x') \frac{1}{Z}Q(x')\\
	= \sum_{x'} P(X_{t+1} = x'| X_t = x) \frac{1}{Z}Q(x)\\
	= \frac{1}{Z}Q(x)$\\
\end{comment}

\textbf{Fund Th of MC} $\lim _{t \to \infty } q_t = \pi ; \; \forall q_0$

\textbf{Ergodic th} $\frac{1}{n} \sum_i f(x_i) \rightarrow \sum_{x} \pi (x) f(x) = \mathbb{E}_{x \sim \pi} [f(x)]$

\subsection{Metropolis Hastings}
\textbf{1) Proposal:} Given $X_t=x$, sample $x' \sim R(X' | X=x)$\\
\textbf{2) Accept:} w/ probability $\alpha = \min (1, \frac{Q(x')R(x|x')}{Q(x)R(x'|x)})$ (if $\alpha < u$)\\
\textbf{Theorem:} Stationary is $\frac{1}{Z}Q(x)$ $\forall $arbitrary $r$\\
\begin{comment}
	The performance depends heavily on what we choose as proposal distribution R.\\
\end{comment}

\textbf{Gibbs:} $t..\infty: i\sim Uniform,$ update $x_i = P(X_i| x_{-i})$\\
$P(X_i = x_i| x_{-i}) = \frac{1}{Z} Q(X_i = x_i, x_{-i}) = \frac{Q(x_{1:n})}{\sum_{x_i} Q(X_i=x_i, x_{-1})}$\\
$\uparrow$ sat. balance equation, practical variant not. $P(X_i| x_{-i})$ eff.\\
\begin{comment}
	We are only updating one of the variables by conditioning on all the other variables.\\
	This can be done efficiently, as the normalizer has to be computed over all $x_i$, which are limited in f.e. a categorical distribution.\\
\end{comment}

\textbf{LLN (Ergodic):} $lim_{N\rightarrow \infty}
\frac{1}{N} \sum_i^N f(x_i) = \sum_{x \in D} \pi(x)f(x)$\\
$D$ finite state space, $f$ defined on D, stat dist $\pi$, $X_{1:n}$ EMC

\begin{comment}
By design of the Markov Chain, the samples depend on each other, so the law of large numbers and the complexity bounds do not apply.\\
	Special case is when the Markov chain is Ergodic and the state space D is finite with stationary $\pi$.
	This is also called the law of large numbers for Markov Chains.\\
	This talks about if it converges to the correct thing, but unfortunately not how fast it converges.\\
\end{comment}

\textbf{Expectations:} $\mathbb{E}[f(X)] \approx \frac{1}{T-t_0} \sum_{\tau=t_0 +1}^T f(X^{(\tau)})$, $t_0$ burn-in\\
\begin{comment}
	The initial state is very crucial when it comes to time-to-convergence. Often the mode is determined heuristically, then started from there.\\
	Establishing convergence rates is very difficult in general.\\
\end{comment}

\textbf{General RV:} $Q(x) = \frac{1}{Z}\exp(-f(x)))$, $f(x) > 0$\\
$p(\theta| X,y) = \frac{1}{Z} p(\theta) p(y| X, \theta) = \frac{1}{Z} \exp(\text{-} [\text{-}\log p(\theta) - \log p(y|X,\theta)])$

\textbf{Bay.Logistic Reg.:} $f(\theta) = \lambda||\theta||^2 \text{+} \sum log(1 \text{+} exp(- y_i \theta^T x_i))$\\

\textbf{Hastings:} $\alpha = \min (1, \frac{R(x|x')}{R(x'|x)}\exp(f(x) - f(x')))$ $| \downarrow$ b.c. sym.\\
$R(x'|x) = \cN(x';x;\tau I) \Rightarrow \alpha = \min \{ 1, \exp(f(x) - f(x')) \}$\\
\begin{comment}
	The R-fraction is 1, since Gaussians are symmetric.\\
	$f(x') < f(x)$: we accept with probability 1 if the cost function is smaller at x'.\\
	$f(x') > f(x)$: cost of new position is higher, still there is a possibility to go into this direction if not too high.\\
	\textbf{Note:} The Algo could get stuck.\\
\end{comment}

$R(.|.) = \cN(x';x \text{-} \tau \nabla f(x);2\tau I)$; if log-concave dist mix t poly(n)\\
\begin{comment}
	MALA incorporates Langevin dynamics. We sample R in direction of the Mode of the energy function, this can help with the direction.\\
	Gradient computation can be expensive, there is a version where stochastic gradients are used.\\
\end{comment}

\textbf{SGLD}
algo; \textbf{1} $i_{1:m} \sim U[{1:n}]$ \textbf{2} $\epsilon{t} \sim \mathcal{N}(0, 2\eta_t I)$ \textbf{3} $\theta_{t+1} = \theta_t + \eta_t (\nabla \log p(\theta_t) + \frac{n}{m} \sum_j^m \nabla log p(y_{i_j} | \theta_t x_{i_j})) + \epsilon_t$; $\eta_t \in O(t^{-1/3})$